%********************************
\chapter{Grundlagen}
\label{chap2:grundlagen}
%********************************
%********************************
\section{Stand der Technik}
\label{sec:sdt}
%********************************
Was sind die neusten Erkenntnisse auf dem Gebiet?

%********************************
\section{Notwendiges Wissen}
\label{sec:wissen}
%********************************
Hier sollen die jeweiligen Grundlagen zu den verwendeten Verfahren/Analysen/Prozesse erläutert werden, so dass der Leser das nötige Wissen besitzt, den Rest der Arbeit zu verstehen.



%********************************
\subsection{Atmospäre}
\label{subsec:atmosphere}

Da das Fluggerät sich in der Atmosphäre bewegt, ist es wichtig die physikalischen Größen zu kennen, welche die aerodynamischen Kräfte und das Triebwerk beeinflussen. Diese sind:
\begin{itemize}
\item{Luftdichte $\rho$}
\item{Temperatur $T$ bzw. Schallgeschwindigkeit $a$}
\item{Druck $p$}
\item{dynamische Viskosität (Zähigkeit) $\eta$}
\end{itemize}



\subsubsection{Reynoldszahls}
\label{rezahl}

Die Reynoldszahl ist eine dimensionslose Kenngröße. Sie stellt das Verhältnis zwischen Trägheits- und Zähigkeitskräften in einem Fluid dar.

\begin{equation}
Re = \frac{\rho * V * l}{\eta} = \frac{V * l}{\nu}
\end{equation}

%********************************
\subsection{Aerodynamische Kräfte}
\label{aerokräfte}

\subsubsection{Beiwerte}
\label{beiwerte}

Um die aerodynamischen Kräfte und Momente einfacher vergleichen zu können werden dimensionslose Größen, sogenannte Beiwerte, verwendet. Der Index kennzeichnet die dazugehörige Kraft oder das Moment. Auftriebsbeiwert $C_A$, Widerstandsbeiwert $C_W$ und Momentenbeiwert $C_M$ werden wie folgt definiert:
\begin{equation}
A = C_A * q * S
\end{equation}
\begin{equation}
W = C_W * q * S
\end{equation}
\begin{equation}
M = C_M * q * S * l_\mu
\end{equation}
\\
Der Ausdruck $S$ bezeichnet dabei die Bezugs(flügel)fläche. Weitere Kraft- und Momentenbeiwerte werden analog definiert.\\

Wenn der jeweilige Beiwert sich nicht global auf den gesamten Flügel, sondern auf eine lokale Stelle bezieht, wird der Index klein geschrieben:

\begin{equation}
dA = C_a * q * l * dy
\end{equation}
\begin{equation}
dW = C_w * q * l * dy
\end{equation}
\begin{equation}
dM = C_m * q * l^2 * dy
\end{equation}
\\
$q = (\rho/2) * V^2$ bezeichnet dabei den Staudruck. Dieser ist von $\rho$ und der Anströmgeschwindigkeit $V$ abhängig.

\subsubsection{Auftrieb und Widerstand}
\label{auftriebwiderstand}

Die senkrecht zur Anströmung wirkende Kraft ist als Auftrieb $A$ definiert.

Die parallel zur Anströmung wirkende Kraft ist als Widerstand $W$ definiert.

Der Nullwiderstandsbeiwert $C_{W0}$ ist der Widerstandsbeiwert bei einem Auftriebsbeiwert von $C_A=0$.

Der induzierte Widerstand entsteht durch ???. Dessen Widerstandsbeiwert kann wie folgt modelliert werden:

\begin{equation}
C_{Wi} = k_i * C^2_A
\end{equation}

Der Widerstandsfaktor $k_i$ hängt von der Flügelstreckung $\Lambda = \frac{b^2}{S}$ und der Auftriebsverteilung in Spannweitenrichtung ab. Ein Minimalwert wird bei einer elliptischen Auftriebsverteilung erreicht (noch umschreiben!!!ZitatHdLfzt, Spannweite $b$ noch einführen!):

\begin{equation}
k_i = \frac{1}{\pi * \Lambda}
\end{equation}

Der Profilwiderstandsfaktorsbeiwert ist abhängig vom Auftrieb. Der zugehörige Widerstandsfaktor ist $k_P$. Bei gewölbten Profilen ist der Auftriebsbeiwert dabei nicht symmetrisch. Für eine spätere analytische Betrachtung kann von folgender Formel ausgegangen werden:

\begin{equation}
C_{WP,A} = k_P * (C_A - C_{AP,0})^2
\end{equation}

Diese Formeln können zusammengefasst und in einer sogenannten Flugzeugpolare dargestellt werden werden:

\begin{equation}
C_{W} = C_{W,min} + k * (C_A - C_{A0})^2
\end{equation}

Geht man von der Annahme einer symmetrischen Polare aus, ergibt sich folgende Gleichung:

\begin{equation}
C_{W} = C_{W0} + k * C^2_A
\end{equation}

Dabei gilt: $C_{A0} = 0$.\\

Die Flugzeugpolare ist abhängig von der Reynolds- und der Machzahl. In der vorliegenden Arbeit wird der Einfluss letzterer allerdings vernachlässigt, da wir uns ausschließlich im niedrigen Unterschallbereich ($Ma < 0,7$) bewegen.

%********************************

\subsection{Stationäre Flugzustände}
\label{stationär}

Bei der Betrachtung stationärer Flugzustände wird das Flugzeug als Punktmasse angesehen. So kann ausschließlich mit den Kraftgleichungen gearbeitet werden.

\subsubsection{Gleitflug}
\label{gleitflug}

Als Gleitflug wird der Flug ohne Antrieb bezeichnet.
Das Kräftegleichgewicht in horizontaler und vertikaler Richtung lautet wie folgt:

\begin{equation}
W = -m * g * sin(\gamma)
\end{equation}

\begin{equation}
A = m * g * cos(\gamma)
\end{equation}

Dabei ist $\gamma$ der Bahnneigungswinkel. Dieser ist bei einem Sinkflug negativ definiert.

Aus den beiden vorangegangenen Gleichungen ergibt sich:

\begin{equation}
tan(\gamma) = -C_W/C_A
\end{equation}

Der Betrag dieses Terms ist als Gleitzahl $\epsilon$ definiert.

\begin{equation}
\epsilon = C_W/C_A
\end{equation}

Die Sinkgeschwindigkeit $w$ ist die vertikale Geschwindigkeitskomponente und ergibt sich zu:

\begin{equation}
w = -V * sin(\gamma)
\end{equation}

\begin{equation}
w = \sqrt{{\frac{2 * m * g}{\rho * S}}} * \frac{C_W}{(C^2_A + C^2_W)^{3/4}}
\end{equation}

Die horizontale Geschwindigkeit $u$ ergibt sich zu:

\begin{equation}
u = \sqrt{{\frac{2 * m * g}{\rho * S}}} * \frac{C_A}{(C^2_A + C^2_W)^{3/4}}
\end{equation}

\subsubsection{Bestes Gleiten}
\label{bestesgleiten}

Der Punkt des besten Gleitens ergibt sich, wenn $\epsilon$ minimal wird. Dazu legt man eine Gerade durch den Ursprung tangential an die Polare an. Für eine symmetrische Polare ergeben sich folgende Zusammenhänge:

\begin{equation}
\epsilon_{min} = 2 * \sqrt{C_{W0} * k}
\end{equation}

\begin{equation}
C^*_A = \sqrt{C_{W0} / k}
\end{equation}

\begin{equation}
C^*_W = 2 * C_{W0}
\end{equation}

mit $\epsilon^2_{min} << 1$ gilt:

\begin{equation}
V^* = \sqrt{\frac{2 * m * g}{C^*_A * \rho * S}}
\end{equation}

\subsubsection{Horizontalflug}
\label{horizontalflug}

Im Horizontalflug ($\gamma = 0$) muss der Auftrieb die Gewichtskraft und die Schubkraft den Widerstand kompensieren.

\begin{equation}
F = W
\end{equation}

\begin{equation}
A = m * g
\end{equation}

Die Fluggeschwindigkeit ergibt sich zu:

\begin{equation}
V = \sqrt{\frac{2 * m * g}{C_A * \rho * S}}
\end{equation}

Die Minimalgeschwindigkeit $V_{min}$ ergibt sich daher zu:

\begin{equation}
V_{min} = \sqrt{\frac{2 * m * g}{C_{A,max} * \rho * S}}
\end{equation}

Mit einer symmetrischen Flugzeugpolare ergibt sich der Widerstand im Horizontalflug zu:
\begin{equation}
W = C_{W0} * \rho/2 * V^2 * S + k * \frac{(m*g)^2}{(\rho/2 * V^2 * S)}
\end{equation}

Mit voriger Gleichung ... ergibt sich der Minimalwiderstand zu:

\begin{equation}
W_{min} = \epsilon_{min} * m * g
\end{equation}

\subsubsection{Steigflug}
\label{steigflug}

Das Kräftegleichgewicht stellt sich folgendermaßen auf:

\begin{equation}
F = W + m * g * sin(\gamma)
\end{equation}

\begin{equation}
A = m * g * cos(\gamma)
\end{equation}

Mit $A = q * C_A * S$ und $W = q * C_W * S$ folgt aus der ersten beider Gleichungen / Aus der ersten Gleichung folgt:

\begin{equation}
sin(\gamma) = \frac{F}{m * g} - \frac{C_W * q * S}{m * g} = \frac{F - W}{m * g}
\end{equation}

Für kleine Winkel $\gamma$ kann man annehmen, das $A = m * g$ gilt. Daraus folgt:

\begin{equation}
sin(\gamma) = \frac{F}{m * g} - \frac{W}{A} = \frac{F}{m * g} - \frac{C_W}{C_A} = \frac{F}{m * g} - \epsilon
\end{equation}

Die Steiggeschwindigkeit $-w$ entspricht der vertikalen Komponente der Geschwindigkeit $V_v$, welche auch als "Specific Excess Power" SEP bezeichnet wird. Diese ist:

\begin{equation}
V_v = -w = V * sin(\gamma) = V * \frac{F}{m * g} - \epsilon
\end{equation}

\subsubsection{Horizontaler Kurvenflug}
\label{horizontalerkurvenflug}

Um eine Richtungsänderung um einen Azimutwinkel $\chi$ durchzuführen, wird im Kurvenflug der Auftriebsvektor um einen Rollwinkel $\phi$ geneigt.

Das Kräftegleichgewicht stellt sich wie folgt auf:

\begin{equation}
F = W
\end{equation}

\begin{equation}
m * g = A * cos(\phi)
\end{equation}

\begin{equation}
m * V * \dot{\chi} = A * sin(\phi)
\end{equation}

mit $\dot{\chi} = V / r_K$ folgt:

\begin{equation}
\frac{m * V^2}{r_K} = A * sin(\phi)
\end{equation}

Aus Gleichung ... und ... folgt der Zusammenhang zwischen $\phi$, $\dot{\chi}$ und $V$:

\begin{equation}
tan(\phi) = \frac{V}{g} * \dot{\chi} = \frac{V^2}{g * r_K}
\end{equation}

Der Lastfaktor $n$ beschreibt das Verhältnis zwischen Auftrieb und Gewicht:

\begin{equation}
n = \frac{A}{m * g} = \frac{1}{cos(\phi)}
\end{equation}

Wie aus den Formeln erkennbar, muss $C_A$ in der Kurve größer sein, als im Horizontalflug. Der Auftriebsbeiwert und mit ihm der Widerstandsbeiwert wächst wie folgt an:

\begin{equation}
C_A = \frac{2 * A}{\rho * V^2 * S} = \frac{2* n * m * g}{\rho * V^2 * S}
\end{equation}

Der Kurvenradius in Abhängigkeit von $n$ ergibt sich zu:

\begin{equation}
r_K = \frac{V^2 / g}{\sqrt{n^2 - 1}}
\end{equation}

Zu einem vorgegebenen $n_{max}$ kann somit $r_{K,min}$ ermittelt werden.

Die Wendegeschwindigkeit $\dot{\chi}$ sollte für möglichst schnelle Wenden maximal werden:

\begin{equation}
\dot{\chi} = V / r_K
\end{equation}

%*******************************

\subsection{Start und Landung}
\label{startlandung}

\subsubsection{Start}
\label{start}

Für die Rollstrecke ergeben sich folgende Kräftegleichgewichte:

\begin{equation}
m * \dot{V} = F - W_R - \mu_R * N_H - \mu * N_B
\end{equation}

\begin{equation}
m * g = N_B + N_H + A_R
\end{equation}

\begin{equation}
N_B * (x_B - \mu_R * z_H) - N_H * (x_H + \mu_R * z_H) - F * z_F - A_R * x_A + \Delta M_\eta = 0
\end{equation}

Die Rollstrecke kann integriert werden über:

\begin{equation}
\frac{dx}{dV} = \frac{m * V}{F - W_R - \mu_R * (m * g - A_R)}
\end{equation}

Hier noch Abschätzformel aus Handbuch der Luftfahrzeugtechnik unter Annahme $C_A$ und $C_W$ konstant während des Startvorgangs.

\subsubsection{Landung}
\label{landung}

Rollstrecke ähnlich Startrollstrecke.

%********************************

\subsection{Leitwerksauslegung}
\label{leitwerk}

Momentengleichgewicht

\subsection{Stabilität}
\label{stabilität}

Der Neutralpunkt ist der Punkt, an dem $\frac{dC_M}{d\alpha} = 0$ ist. Eine Störung des Anstellwinkels wird an diesem Punkt nicht reagiert. Es ist aber ein rückstellendes Moment gewünscht um statische Stabilität zu erreichen. Das ist gegeben, wenn $\frac{dC_M}{d\alpha} < 0$ ist. Dazu muss Schwerpunkt vor dem Neutralpunkt liegen.

Das Stabilitätsmaß $\frac{x_N - x_S}{l_\mu}$ ist eine Größe die zum Vergleich herangezogen werden kann. Richtwerte hierfür sind 5 - 15%.



%********************************

%********************************
\newpage
Normatmossphäre

Aerodynamisches und Geodätisches Koordinatensystem

Rein inkompressible Strömung

Aerodynamische Beiwerte

Laminare, turbulente Grenzschicht

Induzierter Widerstand

Re-Zahl

Grundgleichungen Flugbahn

Luftraum

Schub sehr einfach, da Windkanalwerte verwendet werden.

Hauptsächlich stationäre Betrachtungen

Steigen:

Kurven: Wendedauer

