\selectlanguage{ngerman}
%********************************
\chapter{Grundlagen}
\label{chap2:grundlagen}
%********************************
%********************************
\section{Stand der Technik}
\label{sec:sdt}
%********************************
Was sind die neusten Erkenntnisse auf dem Gebiet?

%********************************
% Schreiben von unterschieden, Motorisiert/Segelflug
\section{Unterschiede zwischen Seglern und motorisierten Flugzeugen}

Einen Teil im Modellflug macht das Segelfliegen aus. Diese haben einige Vorteile gegenüber motorisierten Fliegern. Jene sparen Gewicht, da kein Motor und nur eine kleinere Batterie benötigt wird. Dafür können diese nur 


\section[short]{Aufgaben}

\subsection*{Startstrecke}
Dieses Trifft nur zu, wenn das Flugzeug aus eigener Kraft starten soll und nicht geworfen oder geschleppt wird.

Je nach Gelände auf dem Geflogen werden soll, kann die Länge der Startstrecke von Relevanz sein. Probleme stellen dabei die Unterschiedlichen Rollwiederständen verschiedener Untergründe. Dabei hat Asphalt einen deutlich niedrigeren als hohes Gras. Für einen schnellen Start ist daher 

\section{Air Cargo Challenge}

Im Folgenden werden einige Grundlagen erklärt. Dieses wird anhand des Beispiels ACC2022 getan.

\subsection{Konstruktionsbeschränkungen}

Im Modellflug ist die Größe des Modellflugzeuges hauptsächlich durch die EU-Drohnenverordnung geregelt. Dadurch dürfen nur Drohnen unter 25 kg geflogen werden, solange kein EU-Fernpiloten-Zeugnis vorhanden ist. Zusätzlich werden im ACC Regelwerk noch weitere Beschränkungen geregelt. Der Motor sowie die Batterie sind vorgegeben und das Flugzeug muss in eine Box mit den Seitenlängen $1.5\;\text(m)$ passen und eine maximal höhe von $1$ nicht überschreiten


\subsection{Flugaufgabe}

Für jedes Modellflugzeug muss eine Flugaufgabe definiert werden, auf die optimiert werden kann. Für die ACC muss zunächst eine möglichst kurze Startstrecke von 100 Meter bzw 50 Meter erreicht werden.