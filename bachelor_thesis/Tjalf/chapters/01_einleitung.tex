\selectlanguage{ngerman}                % Set language to german
%********************************
\chapter{Einleitung}					% Kapitel
\label{chap:einleitung}					% Querverweis auf "einleitung"kann aufgerufen werden
%********************************		% mit: \ref{chap:einleitung}

%********************************
\section{Motivation}					% Sektion
\label{sec:motivation}
%********************************
Hier erfolgt die Hinführung zum Thema. Worum geht es? Was ist der Sinn und Zweck dieser Arbeit? Dem Leser wird hier auch erklärt, was ihn wo in dieser Arbeit erwartet.

Diese Bachelorarbeit ist entsprungen, um in der Akamodell Stuttgart effizienter Modellflugzeuge auslegen zu können. Die Akamodell Stuttgart ist ein eingetragener Verein an der Universität Stuttgart, an dem Studenten lernen Modellflugzeuge zu entwerfen und zu bauen. Dazu nimmt die Akamodell zwei-jährlich an der Air Cargo Challenge (ACC) teil. Die ACC ist ein Wettbewerb, bei dem verschiedene Teams aus europäischen und einigen internationalen Universitäten gegeneinander antreten, um ein Flugzeug zu konstruieren, dass dem Regelwerk entsprechend die Flugaufgaben möglichst gut zu erfüllen. Besonders für diese Wettbewerbe ist eine schnelle und qualitativ hochwertige Auslegung von Bedeutung. Da solche Auslegungen viel Know-how erfordern, ist es das Ziel ein Tool zu entwickeln, dass dieses vereinfacht.

\subsection{Motivation genau}			% Untersektion
\label{subsec:motgenau}

Im Speziellen befasst sich diese Arbeit mit der Massenabschätzung des Flugzeuges.

%********************************
\section{Zielsetzung}
\label{sec:zielsetzung}
%********************************
In dieser Arbeit soll ein Algorithmus entwickelt werden, der es ermöglicht  in jedem Schritt der Modellflugzeugauslegung 

%********************************
\section{Gliederung der Arbeit}
\label{sec:gliederung}
%********************************
In Folgenden werden zunächst die Grundlagen und Grundlagen erklärt. Danach wird der Stand der Technik der Flugzeugauslegung erklärt und der Bau und Aufbau von Modellflugzeugen erläutert. Danach 
