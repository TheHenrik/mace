\begin{table}[H]
	\begin{tabular}{ll}
		\hline
		\textbf{Symbol}\phantom{123456789}& \textbf{Bedeutung} \phantom{123456789123456789123456789123456789}\\ 
		\hline
		\\
		Kräfte\\
		\hline
		$A$					& Auftrieb\\
		$F$					& Schub\\
		$W$					& Widerstand\\
		$Y$					& Seitenkraft\\
		$R$					& Reibung\\
		$F_{\ddot{U}berschuss}$ & Überschussleistung\\
		$W_{min}$			& minimaler Widerstand\\
		\\
		Momente\\
		\hline
		$L$					& Rollmoment\\
		$M$					& Nickmoment\\
		$N$					& Giermoment\\
		\\
		Aerodynamische Beiwerte\\
		\hline
		$C_{A}$				& Auftriebsbeiwert\\
		$C_{W}$				& Widerstandsbeiwert\\
		$C_{M}$				& Nickmomentenbeiwert\\
		$C_{A,max}$			& maximaler Auftriebsbeiwert\\
		$C_{A,roll}$		& Auftriebsbeiwert beim Rollvorgang\\
		$C_{AP,0}$			& Profilauftriebsbeiwert bei $\alpha = 0$\\
		$C_{A0}$			& Auftriebsbeiwert bei $\alpha = 0$\\
		$C^*_A$				& Auftriebsbeiwert im Punkt des besten Gleitens\\
		$C_{W,roll}$		& Widerstandsbeiwert beim Rollvorgang\\
		$C_{W,profil}$		& Widerstandsbeiwert des Profils\\
		$C_{W0}$			& Nullwiderstandsbeiwert (Widerstandsbeiwert bei $C_A=0$)\\
		$C_{Wi}$			& induzierter Widerstandsbeiwert\\
		$C_{W,min}$			& minimaler Widerstandsbeiwert\\
		$C_{WP,A}$			& auftriebsabhängiger Profilwiderstandsbeiwert\\
		$C^*_W$				& Widerstandsbeiwert im Punkt des besten Gleitens\\
%		$C_{L}$				& Rollmomentenbeiwert\\
%		$C_{N}$				& Giermomentenbeiwert\\
%		$C_{Y}$				& Seitenkraftbeiwert\\
		
		$C_{a}$				& Auftriebsbeiwert des Profils\\
		$C_{w}$				& Widerstandsbeiwert des Profils\\
		$C_{m}$				& Nickmomentenbeiwert des Profils\\
%		$C_{l}$				& Rollmomentenbeiwert des Profils\\
%		$C_{n}$				& Giermomentenbeiwert des Profils\\
%		$C_{y}$				& Seitenkraftbeiwert des Profils\\
		\\
		Geschwindigkeiten\\
		\hline
		$V$					& Geschwindigkeit\\
		$V_{min}$			& Minimalgeschwindigkeit\\
		$V_{Start}$			& Abhebegeschwindigkeit\\
		$V^*$				& Geschwindigkeit im Punkt des besten Gleitens\\
		$u$					& horizontale Geschwindigkeit (in x-Richtung)\\
		$V_h$				& horizontale Gewchwindigkeitskomponente\\
%		$v$					& Geschwindigkeit in y-Richtung\\
		$w$					& Sinkgeschwindigkeit\\
		$V_v$				& Steiggeschwindigkeit / vertikale Geschwindigkeitskomponente / SEP\\
		$V_K$				& Kurvengeschwindigkeit\\
		$V_{K,min}$			& minimaler Kurvengeschwindigkeit\\
		$a$					& Schallgeschwindigkeit\\
		$Ma$				& Machzahl\\
		$p$					& Drehgeschwindigkeit um die Rollachse\\
		$q$					& Drehgeschwindigkeit um die Nickachse\\
		$r$					& Drehgeschwindigkeit um die Gierachse\\
		\\
	\end{tabular}
\end{table}
\begin{table}[H]
	\begin{tabular}{ll}
		Geometrische Größen\\
		\hline
		$S$					& Bezugs(flügel)fläche\\
		$l_\mu$				& Bezugsflügeltiefe\\
		$l$					& Profiltiefe, Länge\\
		$b$					& Spannweite\\
		$\Lambda$			& Flügelstreckung\\
		$\lambda$			& Zuspitzung\\
%		$s$					& Halbspannweite\\
		$r_K$				& Kurvenradius\\
		$r_{K,min}$			& minimaler Kurvenradius\\
		$s$					& Flugstrecke\\
		$h$					& Höhe\\
		\\
		Aerodynamische Größen\\
		\hline
		$Re$				& Reynoldszahl\\
		$\alpha$			& Anstellwinkel\\
		$\epsilon$			& Gleitzahl\\
		$\gamma$			& Bahnneigungswinkel\\
		$\chi$				& Azimutwinkel\\
		$\phi$				& Rollwinkel\\
		$\rho$				& Luftdichte\\
		$T$					& Temperatur\\
		$\eta$				& dynamische Viskosität\\
		$\nu$				& kinematische Viskosität\\
		$p$					& Druck\\
		$q$					& Staudruck\\
		$k$					& Widerstandsfaktor\\
		$k_i$				& Widerstandsfaktor induzierter Widerstand\\
		$k_P$				& Widerstandsfaktor auftriebsabhängiger Profilwiderstand\\
		$\Phi_A$			& Einflussfaktor des Bodeneffekts auf den Auftrieb\\
		$\Phi_W$			& Einflussfaktor des Bodeneffekts auf den Widerstand\\
		$\beta_A$			& Faktor für $\Phi_A$\\
		$\beta_W$			& Faktor für $\Phi_W$\\
		$\delta_A$			& Faktor für $\Phi_A$\\
		$\delta_W$			& Faktor für $\Phi_W$\\
		\\
		Weitere Größen\\
		\hline
		$m$					& Masse\\
		$g$					& Gewichtskonstante\\
		$n$					& Lastvielfaches\\
		$n_{max}$			& maximales Lastvielvaches\\
		$t$					& Zeit\\
		$\mu_{roll}$		& Rollreibungskoeffizient\\
%		$\dot{m}$			& Massendurchsatz\\
	\end{tabular}
\end{table}




\begin{table}[H]
	\begin{tabular}{ll}
		\hline
		Symbol\phantom{123456789}& Bedeutung \phantom{123456789123456789123456789123456789}\\ 
		\hline
		$\boldsymbol{A}$	& Prozessmatrix eines Systems / Beschleunigungsvektor\\
		$\boldsymbol{B}$	& Eingangsmatrix eines Systems\\
		$\boldsymbol{C}$	& Beobachtungsmatrix eines Systems\\
		$\boldsymbol{H}$	& Approximierte Messmatrix\\
		$\boldsymbol{P}$	& Schätzfehlerkovarianzmatrix\\
		$\boldsymbol{Q}$	& Prozessrauschkovarianzmatrix\\
		$\boldsymbol{R}$	& Messrauschkovarianzmatrix / Sichtlinie\\
		$\dot{\boldsymbol{R}}$	& Relativgeschwindigkeit\\
		$\ddot{\boldsymbol{R}}$	& Relativbeschleunigung\\
		$\boldsymbol{S}$	& Innovationskovarianzmatrix\\
		$\boldsymbol{T}_{\alpha}$	& Drehmatrix\\
		$\boldsymbol{Z}$	& Filterinnovation\\
		$\boldsymbol{\phi}$	& Jacobi-Matrix\\
		$\boldsymbol{\Pi}$	& Übergangsmatrix\\
		$M$					& Anzahl Monte-Carlo-Simulationen\\
		$N$					& Navigationskonstante / Filteranzahl\\
	\end{tabular}
\end{table}

\begin{table}[H]
	\begin{tabular}{ll}
		\hline
		Abkürzung\phantom{123456}& Bedeutung \phantom{1234567891234567891234567891234567899}\\ 
		\hline
	\end{tabular}
\end{table}
\begin{addmargin}[0.2cm]{0cm}
\vspace{-0.6cm}
\begin{acronym}[MMAE~~~~~~~~~~~~]
	\setlength{\itemsep}{-\parsep} 
	\acro{APN}{Augmented Proportional Navigation}
	\acro{CA}{Constant Acceleration}
	\acro{CV}{Constant Velocity}
	\acro{CM}{Cruise Missiles}
	\acro{CT}{Constant Turn}
\end{acronym}
\end{addmargin}