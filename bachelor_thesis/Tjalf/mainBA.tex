%\documentclass[a4paper,10pt]{scrreprt} %Dokumentklasse
%***********************************************************************************************%
%===============================================================================================%
% 																								%
%									    Bachelorarbeit zum Thema  								%
% 																								%
%  									   "Beispiel Abschlussarbeit" 								%
% 																								%
%									    von Gregor Zwickl, 2023									%
% 																								%
%===============================================================================================%
%***********************************************************************************************%


% ============= Klassendefinition ============= %

\documentclass[
	%draft, 		% Entwurfsstadium
	final, 			% fertiges Dokument
	paper=a4, 		% Papier Einstellungen
	pagesize=auto,	% Ist bereits default
	fontsize=12pt, 	% Schriftgröße
	ngerman, 		% Sprache
	openright,		% Neue chapter starten auf ungeraden(odd) Seitenzahlen, Alternative: open=right
	%twoside,		% Aktivieren, falls doppelseitiger Druck
	numbers=noendperiod,	% kein Punkt nach Nummerierung, eventuell autoendperiod
]{scrreprt}

% ============ Dokumentinformationen ============ %

\usepackage[
	pdftitle={Name Abschlussarbeit},		% Titel der Arbeit
	pdfsubject={Bachelorarbeit},			% Art der Abschlussarbeit, um was geht es
	pdfauthor={Tjalf Stadel},				% Name des Autors
	pdfkeywords={},							% Stichwörter
	pdfborder={0 0 0},						% {horizontalCornerRadius verticalCornerRadius borderWidth}
	colorlinks=true,						% Links sind farbig
	%breaklinks=true,						% unnötig
	linktocpage=true,						% Auch Seitenzahlen im Inhaltsverzeichnis können angeklickt 												% werden.
	citecolor=black,						% Farbe der Zitate
	menucolor=black,						% Farbe des ...
	urlcolor=black,							% Farbe der Links
	linkcolor=black,						% Farbe der Links
]{hyperref}							% mögl. Farben: (red, green, blue, cyan, magenta, yellow, black, white)


\title{Bachelorarbeit}
\author{Tjalf Stadel}
\date{02. Januar 2023}

% ============= Verwendete Pakete ============= %

\usepackage[T1]{fontenc} 				% Ausgabefont; bindet europäische (Sonder-)Zeichen ein
\usepackage[utf8]{inputenc} 			% Einbindung von Umlauten, Sonderzeichen (Standart ist nur ASCII)
% oder: \usepackage[latin1]{inputenc}
\usepackage{graphicx} 					% Einbinden von Bildern (.jpg; .png; .pdf)
%\usepackage[pdftex]{graphicx,color}
\usepackage[ngerman, english]{babel} 			% Deutsche Normen / Deutscher Sprachraum
\usepackage{geometry} 					% Seitenlayout
\usepackage{amsmath} 					% Matheumgebung  / Mathepacket
\usepackage{acronym}					% Abkürzungen erstellen
\usepackage[labelfont=bf]{caption}		% Beschriftungen (Schriftart=bf)
\usepackage{tocloft}					% Kontrolle über Layout (Inhaltsverzeichnis, Bilder, ...)
\usepackage{enumitem}					% Kontrolle über Layout (itemize, enumerate, description)
\usepackage{multirow}					% Zusammenfassen von Zeilen und Spalten in Tabellen
\usepackage{float}						% Defining floating objects like figures and tabels
%%\usepackage[headsepline,automark,nouppercase]{scrpage2}	%stellt vorgefertigte PageStyles zur Verfügung
\usepackage{scrlayer-scrpage}			% Kopf- und Fußzeilen mit KOMA-Script
\usepackage{pdfpages}					% PDF-Dokumente einbinden
\usepackage{tikz}						% Erstellen graphischer Elemente
\usepackage{pgfplots}					% Plots erstellen (normale oder logarithmische Skalierung)
\usepackage{pgfgantt}					% Gantt charts zeichen
\usepackage{pdflscape}					% landscapeorientierung von PDFs
\pgfplotsset{compat=newest}			% Kompatibilität mit neuerster Version (gibt immer wieder Updates)
\usepackage{grffile}					% Erweiterter file name support für Grafiken
\usepackage{subcaption}				% Typeset für sub-captions
\usepackage[nottoc]{tocbibind}			% fügt die Bib automatisch zum Inhaltsverzeichnis hinzu 													% (fügt Inhaltsverzeichnis nicht hinzu)
%\raggedbottom 							% bereits voreingestellt


%\usepackage{paralist} % Aufzälungsarten lassen sich ändern

%\usepackage{fancyhdr} % Kopf- und Fußzeile editierbar (nicht mit KOMA-Skript empfohlen)

%\pagestyle{fancy}	% verwendet package fancyhdr (nicht mit KOMA-Skript empfohlen)


% Einstellungen der Kopf- und Fußzeilen
\ohead{\headmark}									% Oben Rechts etwas angeben
\chead{}											% Keine Ausgabe oben in der Mitte
\cfoot[]{}											% Keine Seitenzahl in der Mitte
\ofoot[\pagemark]{\pagemark}						% Seitenzahlen rechts
\automark{section}									% Sektionen in Kopfzeile (Einseitiger Modus)
\renewcommand*{\headfont}{\normalfont}				% Kein kursiver Text in Kopfzeile
\pagestyle{scrheadings}

% Absätze und Abstände
\renewcommand{\baselinestretch}{1}\normalsize		% Zeilenabstand
\setlength{\parindent}{0pt}							% Kein Einrücken nach Absatz
\geometry{inner=31.5mm,outer=31.5mm,bottom=3cm}		% Abstand zu unterem Seitenrand
\textwidth147mm										% Breite des Textbereichs

%%%%%%%%%%%%%%%%%%%%%%%%%%%%%%%%%%%%%%%%%%%%%%%%%%%%%%%%%%%%%%
% ============= 	 Beginn des Dokuments  	   ============= %
%%%%%%%%%%%%%%%%%%%%%%%%%%%%%%%%%%%%%%%%%%%%%%%%%%%%%%%%%%%%%%




\begin{document}
\pagenumbering{Roman}							% Römische Seitenzahlen bis zum Hauptteil
% ============= Titelblatt ============= %
\pdfbookmark[0]{Titelseite}{title}				% Titelseite im pdf als Bookmark anzeigen
\includepdf[pages={1}]{images/deckblatt.pdf}
\addtocounter{page}{-1}

% ============= Leerseite ============ %
\newpage
\pagestyle{empty}
\phantom{t}										% Schafft Platz für Argument, zeigt dieses aber nicht an
\addtocounter{page}{-1}

% ============= Titelseite =========== %
\newpage
\begin{center}
	\begin{Large}
	\textbf{Titel der Arbeit auf DEUTSCH}\\
	\vspace{1cm}
	\textbf{Titel der Arbeit auf ENGLISCH}
\end{Large}

\vspace{4cm}

Bachelorarbeit\\
von\\
Muster Student

\vspace{4cm}

Durchgef\"uhrt am\\
\vspace{0.5cm}
\textbf{INSTITUT F\"UR FLUGZEUGBAU}\\
\textbf{UNIVERSIT\"AT STUTTGART}\\

\vspace{2cm}

Betreuer: M. Sc. Ruben Czichos

\vspace{2cm}

Stuttgart, im November 2018
\end{center}
\addtocounter{page}{-1}

% ==== Originale Aufgabenstellung ==== %
\includepdf{./images/aufgabenstellung.pdf}
\addtocounter{page}{-1}


% ============= Vorspann ============= %

\chapter*{Kurzfassung}
\addcontentsline{toc}{chapter}{Kurzfassung}	% Fügt Eintrag ins Inhaltsverzeichnis hinzu 
\input{chapters/93_kurzfassung}
\newpage
\chapter*{Abstract}
\addcontentsline{toc}{chapter}{Abstract}
\newpage
%\pagestyle{scrheadings}
\renewcommand*{\chapterheadstartvskip}{\vspace*{2.3\baselineskip}}	% Abstand einstellen
%\pdfbookmark[0]{Inhaltsverzeichnis}{toc}							% Inhaltsverzeichnis im pdf als 																			% Bookmark anzeigen
\renewcommand{\baselinestretch}{1.2}\normalsize						% Zeilenabstand
\tableofcontents													% Inhaltsverzeichnis
\renewcommand{\baselinestretch}{1.25}\normalsize					% Zeilenabstand
\renewcommand*{\chapterheadstartvskip}{\vspace*{2.3\baselineskip}}	% Abstand einstellen
% ============= Symbolverzeichnis ============= %
\chapter*{Nomenklatur}
\addcontentsline{toc}{chapter}{Nomenklatur}
Zur Förderung der Übersichtlichkeit werden in der vorliegenden Arbeit Skalare mit normaler Strichstärke und Vektoren bzw. Matrizen fettgedruckt dargestellt.
\renewcommand{\baselinestretch}{1.1}\normalsize						% Zeilenabstand
\begin{table}[H]
	\begin{tabular}{ll}
		\hline
		\textbf{Symbol}\phantom{123456789}& \textbf{Bedeutung} \phantom{123456789123456789123456789123456789}\\ 
		\hline
		\\
		Kräfte\\
		\hline
		$A$					& Auftrieb\\
		$F$					& Schub\\
		$W$					& Widerstand\\
		$Y$					& Seitenkraft\\
		$R$					& Reibung\\
		$F_{\ddot{U}berschuss}$ & Überschussleistung\\
		$W_{min}$			& minimaler Widerstand\\
		\\
		Momente\\
		\hline
		$L$					& Rollmoment\\
		$M$					& Nickmoment\\
		$N$					& Giermoment\\
		\\
		Aerodynamische Beiwerte\\
		\hline
		$C_{A}$				& Auftriebsbeiwert\\
		$C_{W}$				& Widerstandsbeiwert\\
		$C_{M}$				& Nickmomentenbeiwert\\
		$C_{A,max}$			& maximaler Auftriebsbeiwert\\
		$C_{A,roll}$		& Auftriebsbeiwert beim Rollvorgang\\
		$C_{AP,0}$			& Profilauftriebsbeiwert bei $\alpha = 0$\\
		$C_{A0}$			& Auftriebsbeiwert bei $\alpha = 0$\\
		$C^*_A$				& Auftriebsbeiwert im Punkt des besten Gleitens\\
		$C_{W,roll}$		& Widerstandsbeiwert beim Rollvorgang\\
		$C_{W,profil}$		& Widerstandsbeiwert des Profils\\
		$C_{W0}$			& Nullwiderstandsbeiwert (Widerstandsbeiwert bei $C_A=0$)\\
		$C_{Wi}$			& induzierter Widerstandsbeiwert\\
		$C_{W,min}$			& minimaler Widerstandsbeiwert\\
		$C_{WP,A}$			& auftriebsabhängiger Profilwiderstandsbeiwert\\
		$C^*_W$				& Widerstandsbeiwert im Punkt des besten Gleitens\\
%		$C_{L}$				& Rollmomentenbeiwert\\
%		$C_{N}$				& Giermomentenbeiwert\\
%		$C_{Y}$				& Seitenkraftbeiwert\\
		
		$C_{a}$				& Auftriebsbeiwert des Profils\\
		$C_{w}$				& Widerstandsbeiwert des Profils\\
		$C_{m}$				& Nickmomentenbeiwert des Profils\\
%		$C_{l}$				& Rollmomentenbeiwert des Profils\\
%		$C_{n}$				& Giermomentenbeiwert des Profils\\
%		$C_{y}$				& Seitenkraftbeiwert des Profils\\
		\\
		Geschwindigkeiten\\
		\hline
		$V$					& Geschwindigkeit\\
		$V_{min}$			& Minimalgeschwindigkeit\\
		$V_{Start}$			& Abhebegeschwindigkeit\\
		$V^*$				& Geschwindigkeit im Punkt des besten Gleitens\\
		$u$					& horizontale Geschwindigkeit (in x-Richtung)\\
		$V_h$				& horizontale Gewchwindigkeitskomponente\\
%		$v$					& Geschwindigkeit in y-Richtung\\
		$w$					& Sinkgeschwindigkeit\\
		$V_v$				& Steiggeschwindigkeit / vertikale Geschwindigkeitskomponente / SEP\\
		$V_K$				& Kurvengeschwindigkeit\\
		$V_{K,min}$			& minimaler Kurvengeschwindigkeit\\
		$a$					& Schallgeschwindigkeit\\
		$Ma$				& Machzahl\\
		$p$					& Drehgeschwindigkeit um die Rollachse\\
		$q$					& Drehgeschwindigkeit um die Nickachse\\
		$r$					& Drehgeschwindigkeit um die Gierachse\\
		\\
	\end{tabular}
\end{table}
\begin{table}[H]
	\begin{tabular}{ll}
		Geometrische Größen\\
		\hline
		$S$					& Bezugs(flügel)fläche\\
		$l_\mu$				& Bezugsflügeltiefe\\
		$l$					& Profiltiefe, Länge\\
		$b$					& Spannweite\\
		$\Lambda$			& Flügelstreckung\\
		$\lambda$			& Zuspitzung\\
%		$s$					& Halbspannweite\\
		$r_K$				& Kurvenradius\\
		$r_{K,min}$			& minimaler Kurvenradius\\
		$s$					& Flugstrecke\\
		$h$					& Höhe\\
		\\
		Aerodynamische Größen\\
		\hline
		$Re$				& Reynoldszahl\\
		$\alpha$			& Anstellwinkel\\
		$\epsilon$			& Gleitzahl\\
		$\gamma$			& Bahnneigungswinkel\\
		$\chi$				& Azimutwinkel\\
		$\phi$				& Rollwinkel\\
		$\rho$				& Luftdichte\\
		$T$					& Temperatur\\
		$\eta$				& dynamische Viskosität\\
		$\nu$				& kinematische Viskosität\\
		$p$					& Druck\\
		$q$					& Staudruck\\
		$k$					& Widerstandsfaktor\\
		$k_i$				& Widerstandsfaktor induzierter Widerstand\\
		$k_P$				& Widerstandsfaktor auftriebsabhängiger Profilwiderstand\\
		$\Phi_A$			& Einflussfaktor des Bodeneffekts auf den Auftrieb\\
		$\Phi_W$			& Einflussfaktor des Bodeneffekts auf den Widerstand\\
		$\beta_A$			& Faktor für $\Phi_A$\\
		$\beta_W$			& Faktor für $\Phi_W$\\
		$\delta_A$			& Faktor für $\Phi_A$\\
		$\delta_W$			& Faktor für $\Phi_W$\\
		\\
		Weitere Größen\\
		\hline
		$m$					& Masse\\
		$g$					& Gewichtskonstante\\
		$n$					& Lastvielfaches\\
		$n_{max}$			& maximales Lastvielvaches\\
		$t$					& Zeit\\
		$\mu_{roll}$		& Rollreibungskoeffizient\\
%		$\dot{m}$			& Massendurchsatz\\
	\end{tabular}
\end{table}




\begin{table}[H]
	\begin{tabular}{ll}
		\hline
		Symbol\phantom{123456789}& Bedeutung \phantom{123456789123456789123456789123456789}\\ 
		\hline
		$\boldsymbol{A}$	& Prozessmatrix eines Systems / Beschleunigungsvektor\\
		$\boldsymbol{B}$	& Eingangsmatrix eines Systems\\
		$\boldsymbol{C}$	& Beobachtungsmatrix eines Systems\\
		$\boldsymbol{H}$	& Approximierte Messmatrix\\
		$\boldsymbol{P}$	& Schätzfehlerkovarianzmatrix\\
		$\boldsymbol{Q}$	& Prozessrauschkovarianzmatrix\\
		$\boldsymbol{R}$	& Messrauschkovarianzmatrix / Sichtlinie\\
		$\dot{\boldsymbol{R}}$	& Relativgeschwindigkeit\\
		$\ddot{\boldsymbol{R}}$	& Relativbeschleunigung\\
		$\boldsymbol{S}$	& Innovationskovarianzmatrix\\
		$\boldsymbol{T}_{\alpha}$	& Drehmatrix\\
		$\boldsymbol{Z}$	& Filterinnovation\\
		$\boldsymbol{\phi}$	& Jacobi-Matrix\\
		$\boldsymbol{\Pi}$	& Übergangsmatrix\\
		$M$					& Anzahl Monte-Carlo-Simulationen\\
		$N$					& Navigationskonstante / Filteranzahl\\
	\end{tabular}
\end{table}

\begin{table}[H]
	\begin{tabular}{ll}
		\hline
		Abkürzung\phantom{123456}& Bedeutung \phantom{1234567891234567891234567891234567899}\\ 
		\hline
	\end{tabular}
\end{table}
\begin{addmargin}[0.2cm]{0cm}
\vspace{-0.6cm}
\begin{acronym}[MMAE~~~~~~~~~~~~]
	\setlength{\itemsep}{-\parsep} 
	\acro{APN}{Augmented Proportional Navigation}
	\acro{CA}{Constant Acceleration}
	\acro{CV}{Constant Velocity}
	\acro{CM}{Cruise Missiles}
	\acro{CT}{Constant Turn}
\end{acronym}
\end{addmargin}
\renewcommand{\baselinestretch}{1.25}\normalsize					% Zeilenabstand
%\markleft{Nomenklatur}

\renewcommand*{\listoffigures}{
	\begingroup
	%\tocchapter
	\tocfile{\listfigurename}{lof}
	\endgroup
}

\cleardoublepage
\setlength{\cftfigindent}{0cm}
\renewcommand{\baselinestretch}{1.1}\normalsize						% Zeilenabstand
%\pdfbookmark{Abbildungsverzeichnis}{lof} 
\listoffigures 														% Abbildungsverzeichnis
%\markleft{Abbildungsverzeichnis}

\renewcommand*{\listoftables}{
	\begingroup
	%\tocchapter
	\tocfile{\listtablename}{lot}
	\endgroup
}
%\pdfbookmark[0]{Tabellenverzeichnis}{tbv}							% Tabellenverzeichnis im pdf als Bookmark anzeigen
\setlength{\cfttabindent}{0em}
\listoftables														% Tabellenverzeichnis
\renewcommand{\baselinestretch}{1}\normalsize						% Zeilenabstand

% ============= Hauptteil ============= %

\pagenumbering{arabic}  											% Arabische Zahlen im Hauptteil bis zum Anhang

%********************************
\chapter{Einleitung}					% Kapitel
\label{chap:einleitung}					% Querverweis auf "einleitung"kann aufgerufen werden
%********************************		% mit: \ref{chap:einleitung}

%********************************
\section{Motivation}					% Sektion
\label{sec:motivation}
%********************************
Hier erfolgt die Hinführung zum Thema. Worum geht es? Was ist der Sinn und Zweck dieser Arbeit? Dem Leser wird hier auch erklärt, was ihn wo in dieser Arbeit erwartet.

Die Motivation zu dieser Bachelorarbeit entspringt der Mitarbeit in der Akademischen Modellbaugruppe AkaModell Stuttgart e.V.. Dieser studentische Verein hat in der Vergangenheit mehrmals an der AirCargoChallenge teilgenommen. Bei diesem Wettbewerb muss entsprechend eines sich von Bewerb zu Bewerb ändernden Regulariums ein Flugzeug ausgelegt und gebaut werden, welches dann an einem Flugwettbewerb gegen die Konstruktionen anderer Teams antritt. Auch neben diesem Wettbewerb werden innerhalb dieses Vereins immer wieder neue Modellflugzeuge ausgelegt und gebaut.

\subsection{Motivation genau}			% Untersektion
\label{subsec:motgenau}

%********************************
\section{Zielsetzung}
\label{sec:zielsetzung}
%********************************
Das Ziel der vorliegenden Masterarbeit soll sein...

In der Vergangenheit wurden zur Auslegung der Flugzeuge unterschiedliche Entwicklungstools verwendet. Die Auswertung und Analyse verschiedener Konfigurationen erforderte aber meist viel Handarbeit. Daher entstand der Wunsch nach einem zusammenhängenden Entwurfstool, welches verschiedene Teilbereiche einschließt. Das Ziel dabei ist, dass in Zukunft die Flugzeugauslegung effizienter durchgeführt werden kann. Dass gilt sowohl in zeitlicher als auch in qualitativer Hinsicht. Dazu sollen auch Optimierungstools beitragen. Diese sind aber nicht Inhalt dieser studentischen Arbeit. Vielmehr ist es die Ziel alle notwendigen Grundlagen zu schaffen um zu einem späteren Zeitpunkt Optimierungstools für eine jeweils spezifische Anforderung möglichst schnell entwickeln zu können.
Während eine andere studentische Arbeit aus unserem Verein die Grundlagen für eine Massenabschätzung bereitstellt, hat diese Arbeit zur Absicht die Grundlagen für eine aerodynamische Optimierung zu erarbeiten und zusammenzufassen.
\ref{subsec:motgenau}				% Referenz auf Untersektion

%********************************
\section{Gliederung der Arbeit}
\label{sec:gliederung}
%********************************
Die vorliegende Arbeit beginnt mit...

Es sollen für den gesamten Entwurfsprozess Funktionen bereitgestellt werden. Das heißt es werden einfache Abschätzungsformeln für ein frühes Entwicklungsstadium bereitgestellt werden. Um ein genaueres Feintuning in einem fortgeschrittenen Entwicklungsstadium zu ermöglichen ist die Einbindung von OpenSource-Software angedacht.

Im Rahmen dieser Arbeit wird recherchiert, welche Entwurfsformeln bereits bekannt sind um diese in den Gesamtalgorithmus aufzunehmen. Zusätzlich werden eigene Algorithmen für die Anwendung im in unserem Anforderungsbereich entwickelt. Diese Bachelorarbeit soll also Vorhandenes mit Eigenem in einem neuen Tool verbinden.

Da eine analytische Betrachtung eines gesamten Flugs mit seinen unterschiedlichen Flugzuständen nicht möglich ist, müssen alle Flugzustände getrennt/punktuell betrachtet und bewertet werden. Besonders eignen sich hierzu stationäre Flugzustände, da sie gut analytisch betrachtet werden können.

Ein Flug wird dazu in mehrere Flugabschnitte unterteilt. (Start, Beschleunigung, Steigen, Horizontalflug, Kurvenflug, Sinken, Landung) Im Rahmen dieser Arbeit sollen Tools bereitgestellt werden um diese unterschiedlichen Abschnitte zu betrachten. Diese Abschnitte sind für die meisten Flugzeuge allgemein gültig. Je nach Mission liegen die Schwerpunkte allerdings in unterschiedlichen Bereichen. Mithilfe der Tools können später verschiedene Missionsanforderungen modelliert und Flugzeugauslegungen diesbezüglich analysiert werden.

%********************************
\chapter{Grundlagen}
\label{chap2:grundlagen}
%********************************
%********************************
\section{Stand der Technik}
\label{sec:sdt}
%********************************
Was sind die neusten Erkenntnisse auf dem Gebiet?

%********************************
\section{Notwendiges Wissen}
\label{sec:wissen}
%********************************
Hier sollen die jeweiligen Grundlagen zu den verwendeten Verfahren/Analysen/Prozesse erläutert werden, so dass der Leser das nötige Wissen besitzt, den Rest der Arbeit zu verstehen.



%********************************
\subsection{Atmospäre}
\label{subsec:atmosphere}

Da das Fluggerät sich in der Atmosphäre bewegt, ist es wichtig die physikalischen Größen zu kennen, welche die aerodynamischen Kräfte und das Triebwerk beeinflussen. Diese sind:
\begin{itemize}
\item{Luftdichte $\rho$}
\item{Temperatur $T$ bzw. Schallgeschwindigkeit $a$}
\item{Druck $p$}
\item{dynamische Viskosität (Zähigkeit) $\eta$}
\end{itemize}



\subsubsection{Reynoldszahls}
\label{rezahl}

Die Reynoldszahl ist eine dimensionslose Kenngröße. Sie stellt das Verhältnis zwischen Trägheits- und Zähigkeitskräften in einem Fluid dar.

\begin{equation}
Re = \frac{\rho * V * l}{\eta} = \frac{V * l}{\nu}
\end{equation}

%********************************
\subsection{Aerodynamische Kräfte}
\label{aerokräfte}

\subsubsection{Beiwerte}
\label{beiwerte}

Um die aerodynamischen Kräfte und Momente einfacher vergleichen zu können werden dimensionslose Größen, sogenannte Beiwerte, verwendet. Der Index kennzeichnet die dazugehörige Kraft oder das Moment. Auftriebsbeiwert $C_A$, Widerstandsbeiwert $C_W$ und Momentenbeiwert $C_M$ werden wie folgt definiert:
\begin{equation}
A = C_A * q * S
\end{equation}
\begin{equation}
W = C_W * q * S
\end{equation}
\begin{equation}
M = C_M * q * S * l_\mu
\end{equation}
\\
Der Ausdruck $S$ bezeichnet dabei die Bezugs(flügel)fläche. Weitere Kraft- und Momentenbeiwerte werden analog definiert.\\

Wenn der jeweilige Beiwert sich nicht global auf den gesamten Flügel, sondern auf eine lokale Stelle bezieht, wird der Index klein geschrieben:

\begin{equation}
dA = C_a * q * l * dy
\end{equation}
\begin{equation}
dW = C_w * q * l * dy
\end{equation}
\begin{equation}
dM = C_m * q * l^2 * dy
\end{equation}
\\
$q = (\rho/2) * V^2$ bezeichnet dabei den Staudruck. Dieser ist von $\rho$ und der Anströmgeschwindigkeit $V$ abhängig.

\subsubsection{Auftrieb und Widerstand}
\label{auftriebwiderstand}

Die senkrecht zur Anströmung wirkende Kraft ist als Auftrieb $A$ definiert.

Die parallel zur Anströmung wirkende Kraft ist als Widerstand $W$ definiert.

Der Nullwiderstandsbeiwert $C_{W0}$ ist der Widerstandsbeiwert bei einem Auftriebsbeiwert von $C_A=0$.

Der induzierte Widerstand entsteht durch ???. Dessen Widerstandsbeiwert kann wie folgt modelliert werden:

\begin{equation}
C_{Wi} = k_i * C^2_A
\end{equation}

Der Widerstandsfaktor $k_i$ hängt von der Flügelstreckung $\Lambda = \frac{b^2}{S}$ und der Auftriebsverteilung in Spannweitenrichtung ab. Ein Minimalwert wird bei einer elliptischen Auftriebsverteilung erreicht (noch umschreiben!!!ZitatHdLfzt, Spannweite $b$ noch einführen!):

\begin{equation}
k_i = \frac{1}{\pi * \Lambda}
\end{equation}

Der Profilwiderstandsfaktorsbeiwert ist abhängig vom Auftrieb. Der zugehörige Widerstandsfaktor ist $k_P$. Bei gewölbten Profilen ist der Auftriebsbeiwert dabei nicht symmetrisch. Für eine spätere analytische Betrachtung kann von folgender Formel ausgegangen werden:

\begin{equation}
C_{WP,A} = k_P * (C_A - C_{AP,0})^2
\end{equation}

Diese Formeln können zusammengefasst und in einer sogenannten Flugzeugpolare dargestellt werden werden:

\begin{equation}
C_{W} = C_{W,min} + k * (C_A - C_{A0})^2
\end{equation}

Geht man von der Annahme einer symmetrischen Polare aus, ergibt sich folgende Gleichung:

\begin{equation}
C_{W} = C_{W0} + k * C^2_A
\end{equation}

Dabei gilt: $C_{A0} = 0$.\\

Die Flugzeugpolare ist abhängig von der Reynolds- und der Machzahl. In der vorliegenden Arbeit wird der Einfluss letzterer allerdings vernachlässigt, da wir uns ausschließlich im niedrigen Unterschallbereich ($Ma < 0,7$) bewegen.

%********************************

\subsection{Stationäre Flugzustände}
\label{stationär}

Bei der Betrachtung stationärer Flugzustände wird das Flugzeug als Punktmasse angesehen. So kann ausschließlich mit den Kraftgleichungen gearbeitet werden.

\subsubsection{Gleitflug}
\label{gleitflug}

Als Gleitflug wird der Flug ohne Antrieb bezeichnet.
Das Kräftegleichgewicht in horizontaler und vertikaler Richtung lautet wie folgt:

\begin{equation}
W = -m * g * sin(\gamma)
\end{equation}

\begin{equation}
A = m * g * cos(\gamma)
\end{equation}

Dabei ist $\gamma$ der Bahnneigungswinkel. Dieser ist bei einem Sinkflug negativ definiert.

Aus den beiden vorangegangenen Gleichungen ergibt sich:

\begin{equation}
tan(\gamma) = -C_W/C_A
\end{equation}

Der Betrag dieses Terms ist als Gleitzahl $\epsilon$ definiert.

\begin{equation}
\epsilon = C_W/C_A
\end{equation}

Die Sinkgeschwindigkeit $w$ ist die vertikale Geschwindigkeitskomponente und ergibt sich zu:

\begin{equation}
w = -V * sin(\gamma)
\end{equation}

\begin{equation}
w = \sqrt{{\frac{2 * m * g}{\rho * S}}} * \frac{C_W}{(C^2_A + C^2_W)^{3/4}}
\end{equation}

Die horizontale Geschwindigkeit $u$ ergibt sich zu:

\begin{equation}
u = \sqrt{{\frac{2 * m * g}{\rho * S}}} * \frac{C_A}{(C^2_A + C^2_W)^{3/4}}
\end{equation}

\subsubsection{Bestes Gleiten}
\label{bestesgleiten}

Der Punkt des besten Gleitens ergibt sich, wenn $\epsilon$ minimal wird. Dazu legt man eine Gerade durch den Ursprung tangential an die Polare an. Für eine symmetrische Polare ergeben sich folgende Zusammenhänge:

\begin{equation}
\epsilon_{min} = 2 * \sqrt{C_{W0} * k}
\end{equation}

\begin{equation}
C^*_A = \sqrt{C_{W0} / k}
\end{equation}

\begin{equation}
C^*_W = 2 * C_{W0}
\end{equation}

mit $\epsilon^2_{min} << 1$ gilt:

\begin{equation}
V^* = \sqrt{\frac{2 * m * g}{C^*_A * \rho * S}}
\end{equation}

\subsubsection{Horizontalflug}
\label{horizontalflug}

Im Horizontalflug ($\gamma = 0$) muss der Auftrieb die Gewichtskraft und die Schubkraft den Widerstand kompensieren.

\begin{equation}
F = W
\end{equation}

\begin{equation}
A = m * g
\end{equation}

Die Fluggeschwindigkeit ergibt sich zu:

\begin{equation}
V = \sqrt{\frac{2 * m * g}{C_A * \rho * S}}
\end{equation}

Die Minimalgeschwindigkeit $V_{min}$ ergibt sich daher zu:

\begin{equation}
V_{min} = \sqrt{\frac{2 * m * g}{C_{A,max} * \rho * S}}
\end{equation}

Mit einer symmetrischen Flugzeugpolare ergibt sich der Widerstand im Horizontalflug zu:
\begin{equation}
W = C_{W0} * \rho/2 * V^2 * S + k * \frac{(m*g)^2}{(\rho/2 * V^2 * S)}
\end{equation}

Mit voriger Gleichung ... ergibt sich der Minimalwiderstand zu:

\begin{equation}
W_{min} = \epsilon_{min} * m * g
\end{equation}

\subsubsection{Steigflug}
\label{steigflug}

Das Kräftegleichgewicht stellt sich folgendermaßen auf:

\begin{equation}
F = W + m * g * sin(\gamma)
\end{equation}

\begin{equation}
A = m * g * cos(\gamma)
\end{equation}

Mit $A = q * C_A * S$ und $W = q * C_W * S$ folgt aus der ersten beider Gleichungen / Aus der ersten Gleichung folgt:

\begin{equation}
sin(\gamma) = \frac{F}{m * g} - \frac{C_W * q * S}{m * g} = \frac{F - W}{m * g}
\end{equation}

Für kleine Winkel $\gamma$ kann man annehmen, das $A = m * g$ gilt. Daraus folgt:

\begin{equation}
sin(\gamma) = \frac{F}{m * g} - \frac{W}{A} = \frac{F}{m * g} - \frac{C_W}{C_A} = \frac{F}{m * g} - \epsilon
\end{equation}

Die Steiggeschwindigkeit $-w$ entspricht der vertikalen Komponente der Geschwindigkeit $V_v$, welche auch als "Specific Excess Power" SEP bezeichnet wird. Diese ist:

\begin{equation}
V_v = -w = V * sin(\gamma) = V * \frac{F}{m * g} - \epsilon
\end{equation}

\subsubsection{Horizontaler Kurvenflug}
\label{horizontalerkurvenflug}

Um eine Richtungsänderung um einen Azimutwinkel $\chi$ durchzuführen, wird im Kurvenflug der Auftriebsvektor um einen Rollwinkel $\phi$ geneigt.

Das Kräftegleichgewicht stellt sich wie folgt auf:

\begin{equation}
F = W
\end{equation}

\begin{equation}
m * g = A * cos(\phi)
\end{equation}

\begin{equation}
m * V * \dot{\chi} = A * sin(\phi)
\end{equation}

mit $\dot{\chi} = V / r_K$ folgt:

\begin{equation}
\frac{m * V^2}{r_K} = A * sin(\phi)
\end{equation}

Aus Gleichung ... und ... folgt der Zusammenhang zwischen $\phi$, $\dot{\chi}$ und $V$:

\begin{equation}
tan(\phi) = \frac{V}{g} * \dot{\chi} = \frac{V^2}{g * r_K}
\end{equation}

Der Lastfaktor $n$ beschreibt das Verhältnis zwischen Auftrieb und Gewicht:

\begin{equation}
n = \frac{A}{m * g} = \frac{1}{cos(\phi)}
\end{equation}

Wie aus den Formeln erkennbar, muss $C_A$ in der Kurve größer sein, als im Horizontalflug. Der Auftriebsbeiwert und mit ihm der Widerstandsbeiwert wächst wie folgt an:

\begin{equation}
C_A = \frac{2 * A}{\rho * V^2 * S} = \frac{2* n * m * g}{\rho * V^2 * S}
\end{equation}

Der Kurvenradius in Abhängigkeit von $n$ ergibt sich zu:

\begin{equation}
r_K = \frac{V^2 / g}{\sqrt{n^2 - 1}}
\end{equation}

Zu einem vorgegebenen $n_{max}$ kann somit $r_{K,min}$ ermittelt werden.

Die Wendegeschwindigkeit $\dot{\chi}$ sollte für möglichst schnelle Wenden maximal werden:

\begin{equation}
\dot{\chi} = V / r_K
\end{equation}

%*******************************

\subsection{Start und Landung}
\label{startlandung}

\subsubsection{Start}
\label{start}

Für die Rollstrecke ergeben sich folgende Kräftegleichgewichte:

\begin{equation}
m * \dot{V} = F - W_R - \mu_R * N_H - \mu * N_B
\end{equation}

\begin{equation}
m * g = N_B + N_H + A_R
\end{equation}

\begin{equation}
N_B * (x_B - \mu_R * z_H) - N_H * (x_H + \mu_R * z_H) - F * z_F - A_R * x_A + \Delta M_\eta = 0
\end{equation}

Die Rollstrecke kann integriert werden über:

\begin{equation}
\frac{dx}{dV} = \frac{m * V}{F - W_R - \mu_R * (m * g - A_R)}
\end{equation}

Hier noch Abschätzformel aus Handbuch der Luftfahrzeugtechnik unter Annahme $C_A$ und $C_W$ konstant während des Startvorgangs.

\subsubsection{Landung}
\label{landung}

Rollstrecke ähnlich Startrollstrecke.

%********************************

\subsection{Leitwerksauslegung}
\label{leitwerk}

Momentengleichgewicht

\subsection{Stabilität}
\label{stabilität}

Der Neutralpunkt ist der Punkt, an dem $\frac{dC_M}{d\alpha} = 0$ ist. Eine Störung des Anstellwinkels wird an diesem Punkt nicht reagiert. Es ist aber ein rückstellendes Moment gewünscht um statische Stabilität zu erreichen. Das ist gegeben, wenn $\frac{dC_M}{d\alpha} < 0$ ist. Dazu muss Schwerpunkt vor dem Neutralpunkt liegen.

Das Stabilitätsmaß $\frac{x_N - x_S}{l_\mu}$ ist eine Größe die zum Vergleich herangezogen werden kann. Richtwerte hierfür sind 5 - 15%.



%********************************

%********************************
\newpage
Normatmossphäre

Aerodynamisches und Geodätisches Koordinatensystem

Rein inkompressible Strömung

Aerodynamische Beiwerte

Laminare, turbulente Grenzschicht

Induzierter Widerstand

Re-Zahl

Grundgleichungen Flugbahn

Luftraum

Schub sehr einfach, da Windkanalwerte verwendet werden.

Hauptsächlich stationäre Betrachtungen

Steigen:

Kurven: Wendedauer



\selectlanguage{ngerman}
%********************************
\chapter{Hauptteil}

%Flowchart

\section[short]{Algorithmus}

\subsection[short]{Flugzeugdarstellung}

%UML Diagramm

Das Flugzeug wird als Klasse dargestellt. Dieses hat verschiedene Parameter darunter 

\subsection[short]{Gitternetz}

Um verschiedene Parameter zu bestimmen, muss zunächst ein 3D-Objekt aus dem Flugzeug erstellt werden. Dafür wird bei dem Flügel das Profil auf die jeweiligen Punkte Projeziert und über die erzeugten Profile ein Netz bestehend aus Dreiecken gespannt.

\subsection{Parameterbestimmung}

\subsubsection*{Flügelfläche}
Da das Gitternetz durch Dreiecke dargestellt wird, kann man die Oberfläche als Summe der Flächen der Dreiecke berechnet werden.

$$
A = \sum \frac{|AB \times AC|}{2}
$$

Die Referenzflügelfläche kann über die Außenpunkte 

\subsubsection[short]{Flügelvolumen}

Das Volumen eines Flügelsegments kann über die Summe eines sechstels des Spatproduktes der drei Ecken der Dreiecke bestimmt werden. Dabei ist die Reihenfolge der Punkte so zu wählen, dass der Normalenvektor aus dem Körper zeigt.

$$
V = \sum \frac{(a\times b )\cdot c}{6}
$$

\subsection[short]{Lastenbestimmung}

%vn diagramm 
%sicherheitfaktor

%********************************
\chapter{Ergebnisse}
\label{chap4:ergebnisse}
%********************************
Hier werden alle während der Arbeit gewonnenen Informationen ausführlich dargestellt, diskutiert und interpretiert. Dabei sollte im Bezug zur Ausgangsituation ein Ergebnis/Erkenntnisgewinn bzw. eine Verbesserung/Änderung vorliegen und eine Schlussfolgerung mit Empfehlung gezogen werden. Aufgetretene Diskrepanzen und mögliche Ursachen sollten klar dargestellt werden.

%********************************
\chapter{Zusammmenfassung und Ausblick}
\label{chap:zusammenfassung_und_ausblick}
%********************************
Zum Schluss sollte nochmals die wesentlichen Ergebnisse klar herausgestellt werden. Außerdem kann ein Ausblick für weitere Arbeiten/Untersuchungen in diesem Bereich erfolgen. Umfang beträgt hier ca. 2 Seiten.


% ============= Literaturverzeichnis ============= %
%\renewcommand{\baselinestretch}{1.2}\normalsize						% Zeilenabstand
\clearpage
\bibliographystyle{plaindin}											% Art von Seitenumbruch
\addcontentsline{toc}{chapter}{Literaturverzeichnis}
\bibliography{Literatur}
%\renewcommand{\baselinestretch}{1}\normalsize						% Zeilenabstand
% =============Anhang ============= %
\appendix
\selectlanguage{ngerman}
%********************************
\addchap{Anhang}
\refstepcounter{chapter} 
\label{chap:anhang}
Hier können noch Tabellen, Messprotokolle, Rechnerprotokolle, Konstruktionszeichnungen sowie Programmcodes und ähnliches zur Dokumentation angeheftet werden.

% ============= Nutzungsrechte ========= %
\newpage
\pagestyle{empty}
\input{chapters/98_Nutzungsrechte}

% ============= Leerseite ============= %
\newpage
\pagestyle{empty}
\phantom{t}


%%%%%%%%%%%%%%%%%%%%%%%%%%%%%%%%%%%%%%%%%%%%%%%%%%%%%%%%%%%%%%
% ============= 	  Ende des Dokuments  	   ============= %
%%%%%%%%%%%%%%%%%%%%%%%%%%%%%%%%%%%%%%%%%%%%%%%%%%%%%%%%%%%%%%
\end{document}